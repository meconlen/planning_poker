\documentclass[11pt,a4paper]{article}
\usepackage[utf8]{inputenc}
\usepackage[english]{babel}
\usepackage{amsmath}
\usepackage{amsfonts}
\usepackage{amssymb}
\usepackage{graphicx}
\usepackage{hyperref}
\usepackage{listings}
\usepackage{xcolor}
\usepackage{geometry}
\usepackage{fancyhdr}
\usepackage{titlesec}
\usepackage{array}
\usepackage{booktabs}
\usepackage{longtable}
\usepackage{float}

% Page geometry
\geometry{
    left=2.5cm,
    right=2.5cm,
    top=3cm,
    bottom=3cm,
    headheight=14pt
}

% Header and footer
\pagestyle{fancy}
\fancyhf{}
\fancyhead[L]{Planning Poker System Design}
\fancyhead[R]{v0.8.2}
\fancyfoot[C]{\thepage}

% Code listing style
\definecolor{codegreen}{rgb}{0,0.6,0}
\definecolor{codegray}{rgb}{0.5,0.5,0.5}
\definecolor{codepurple}{rgb}{0.58,0,0.82}
\definecolor{backcolour}{rgb}{0.95,0.95,0.92}

\lstdefinestyle{mystyle}{
    backgroundcolor=\color{backcolour},   
    commentstyle=\color{codegreen},
    keywordstyle=\color{magenta},
    numberstyle=\tiny\color{codegray},
    stringstyle=\color{codepurple},
    basicstyle=\ttfamily\footnotesize,
    breakatwhitespace=false,         
    breaklines=true,                 
    captionpos=b,                    
    keepspaces=true,                 
    numbers=left,                    
    numbersep=5pt,                  
    showspaces=false,                
    showstringspaces=false,
    showtabs=false,                  
    tabsize=2,
    frame=single
}

\lstset{style=mystyle}

% JSON style for message documentation
\lstdefinelanguage{json}{
    basicstyle=\normalfont\ttfamily,
    numbers=left,
    numberstyle=\scriptsize,
    stepnumber=1,
    numbersep=8pt,
    showstringspaces=false,
    breaklines=true,
    frame=lines,
    backgroundcolor=\color{backcolour},
    literate=
     *{0}{{{\color{blue}0}}}{1}
      {1}{{{\color{blue}1}}}{1}
      {2}{{{\color{blue}2}}}{1}
      {3}{{{\color{blue}3}}}{1}
      {4}{{{\color{blue}4}}}{1}
      {5}{{{\color{blue}5}}}{1}
      {6}{{{\color{blue}6}}}{1}
      {7}{{{\color{blue}7}}}{1}
      {8}{{{\color{blue}8}}}{1}
      {9}{{{\color{blue}9}}}{1}
      {:}{{{\color{red}{:}}}}{1}
      {,}{{{\color{red}{,}}}}{1}
      {\{}{{{\color{red}{\{}}}}{1}
      {\}}{{{\color{red}{\}}}}}{1}
      {[}{{{\color{red}{[}}}}{1}
      {]}{{{\color{red}{]}}}}{1},
}

% Hyperref setup
\hypersetup{
    colorlinks=true,
    linkcolor=blue,
    filecolor=magenta,      
    urlcolor=cyan,
    citecolor=green,
    pdftitle={Planning Poker System Design},
    pdfauthor={Mike Conlen},
    pdfsubject={Real-time collaborative estimation system},
    pdfkeywords={planning poker, websockets, go, real-time, scrum, agile}
}

% Title formatting
\titleformat{\section}
{\normalfont\Large\bfseries}{\thesection}{1em}{}
\titleformat{\subsection}
{\normalfont\large\bfseries}{\thesubsection}{1em}{}

\begin{document}

% Title page
\begin{titlepage}
    \centering
    \vspace*{2cm}
    
    {\Huge\bfseries Planning Poker System Design}
    
    \vspace{1cm}
    {\Large Real-time Collaborative Estimation Platform}
    
    \vspace{2cm}
    {\large\textbf{Author:} Mike Conlen}
    
    \vspace{1cm}
    {\large\textbf{Version:} 0.7.5}
    
    \vspace{1cm}
    {\large\textbf{Date:} \today}
    
    \vspace{3cm}
    
    \begin{abstract}
    This document describes the architectural design and implementation of a real-time Planning Poker system built with Go and WebSockets. The system provides a collaborative platform for Agile development teams to perform story point estimation in distributed environments. This specification covers the system architecture, WebSocket message protocols, session management, and security considerations.
    \end{abstract}
    
    \vfill
    
    {\small Planning Poker Project Documentation\\
    \url{https://github.com/meconlen/planning_poker}}
    
\end{titlepage}

\tableofcontents
\newpage

\section{Introduction}

Planning Poker is a consensus-based, gamified technique for estimating effort or relative size of development goals in software development \cite{grenning2002planning}. This system implements a real-time, web-based Planning Poker platform that enables distributed teams to collaborate effectively on story point estimation, following Agile methodologies \cite{schwaber2020scrum}.

\subsection{Purpose and Scope}

The Planning Poker system addresses the need for remote Agile teams to conduct estimation sessions with the same effectiveness as in-person meetings. The system provides:

\begin{itemize}
    \item Real-time collaborative estimation sessions
    \item Moderator-controlled session flow
    \item Standard Fibonacci voting scales
    \item Session state persistence
    \item Multi-user support with role-based permissions
\end{itemize}

\subsection{System Overview}

The system follows a client-server architecture with WebSocket-based real-time communication \cite{rfc6455}. The server is implemented in Go \cite{golang2024}, leveraging the Gorilla WebSocket library \cite{gorilla2024} for efficient bidirectional communication. The client interface is a modern web application using vanilla JavaScript and WebSocket APIs.

\section{System Architecture}

\subsection{High-Level Architecture}

The Planning Poker system employs a three-tier architecture:

\begin{enumerate}
    \item \textbf{Presentation Layer}: Web-based user interface
    \item \textbf{Application Layer}: Go-based server with WebSocket handling
    \item \textbf{Data Layer}: In-memory session management with future persistence options
\end{enumerate}

\subsection{Technology Stack}

\begin{table}[H]
\centering
\begin{tabular}{@{}ll@{}}
\toprule
\textbf{Component} & \textbf{Technology} \\
\midrule
Backend Language & Go 1.24+ \\
WebSocket Library & Gorilla WebSocket v1.5.3 \\
UUID Generation & Google UUID v1.6.0 \\
Frontend & HTML5, CSS3, JavaScript (ES6+) \\
Containerization & Docker \\
CI/CD & GitHub Actions \\
Documentation & \LaTeX \\
\bottomrule
\end{tabular}
\caption{Technology Stack}
\label{tab:tech-stack}
\end{table}

\subsection{Package Structure}

The Go application follows clean architecture principles with clear separation of concerns:

\begin{lstlisting}[language=bash, caption=Project Structure]
planning-poker/
|-- main.go                     # Application entry point
|-- internal/
|   |-- server/
|   |   `-- server.go           # HTTP and WebSocket handlers
|   `-- poker/
|       |-- session.go          # Session management logic
|       |-- session_test.go     # Unit tests
|       `-- session_unit_test.go # Comprehensive test suite
|-- web/
|   `-- index.html              # Frontend application
|-- test/
|   `-- client.go               # Integration test client
|-- scripts/
|   |-- workflow.sh             # Development workflow
|   |-- monitor-actions.sh      # CI/CD monitoring
|   `-- check-actions.sh        # Status checking
`-- docs/
    |-- design.tex              # This document
    |-- references.bib          # Bibliography
    `-- Makefile                # Documentation build
\end{lstlisting}

\section{Session Management}

\subsection{Session Lifecycle}

Sessions progress through distinct states that control user interactions and system behavior:

\begin{enumerate}
    \item \textbf{Waiting}: Initial state where participants join but cannot vote
    \item \textbf{Active}: Session is running and users can participate in voting
    \item \textbf{Completed}: Session has ended (future enhancement)
\end{enumerate}

\subsection{User Roles}

The system implements role-based access control with two primary roles:

\begin{table}[H]
\centering
\begin{tabular}{@{}lp{8cm}@{}}
\toprule
\textbf{Role} & \textbf{Permissions} \\
\midrule
Moderator & 
\begin{minipage}[t]{8cm}
\begin{itemize}
    \item Start and control session flow
    \item Reveal votes
    \item Set story descriptions
    \item Initiate new voting rounds
    \item Manage session state
\end{itemize}
\end{minipage} \\
\midrule
Participant & 
\begin{minipage}[t]{8cm}
\begin{itemize}
    \item Submit votes
    \item View session state
    \item Wait in waiting room until session starts
    \item Receive real-time updates
\end{itemize}
\end{minipage} \\
\bottomrule
\end{tabular}
\caption{User Roles and Permissions}
\label{tab:user-roles}
\end{table}

\section{WebSocket Message Protocol}

\subsection{Message Format}

All WebSocket messages follow a standardized JSON format \cite{rfc7159} for consistency and ease of parsing:

\begin{lstlisting}[language=json, caption=Base Message Format]
{
    "type": "message_type",
    "data": {
        // Message-specific payload
    }
}
\end{lstlisting}

\subsection{Client-to-Server Messages}

\subsubsection{Vote Submission}

Participants submit their estimates using the vote message:

\begin{lstlisting}[language=json, caption=Vote Message]
{
    "type": "vote",
    "data": {
        "vote": "5"
    }
}
\end{lstlisting}

\textbf{Validation:} The vote value must be from the standard Fibonacci sequence: 0, 0.5, 1, 2, 3, 5, 8, 13, 21, ?, coffee

\subsubsection{Session Control}

Moderators control session flow with the following messages:

\begin{lstlisting}[language=json, caption=Start Session Message]
{
    "type": "start_session"
}
\end{lstlisting}

\begin{lstlisting}[language=json, caption=Reveal Votes Message]
{
    "type": "reveal"
}
\end{lstlisting}

\begin{lstlisting}[language=json, caption=New Round Message]
{
    "type": "new_round"
}
\end{lstlisting}

\begin{lstlisting}[language=json, caption=Set Story Message]
{
    "type": "set_story",
    "data": {
        "story": "User story description"
    }
}
\end{lstlisting}

\subsection{Server-to-Client Messages}

\subsubsection{Session State Broadcast}

The server broadcasts complete session state to all connected clients:

\begin{lstlisting}[language=json, caption=Session State Message]
{
    "type": "session_state",
    "data": {
        "status": "active",
        "currentStory": "As a user, I want to...",
        "votesRevealed": false,
        "users": {
            "user-uuid-1": {
                "id": "user-uuid-1",
                "name": "Alice",
                "vote": "5",
                "isModerator": true,
                "isOnline": true
            },
            "user-uuid-2": {
                "id": "user-uuid-2",
                "name": "Bob",
                "vote": null,
                "isModerator": false,
                "isOnline": true
            }
        }
    }
}
\end{lstlisting}

\subsubsection{Waiting Room Notification}

Non-moderator users receive waiting room notifications:

\begin{lstlisting}[language=json, caption=Waiting Room Message]
{
    "type": "waiting_room",
    "data": {
        "message": "Waiting for moderator to start the session..."
    }
}
\end{lstlisting}

\subsubsection{Session Start Notification}

When a session begins, all participants receive:

\begin{lstlisting}[language=json, caption=Session Start Message]
{
    "type": "start_session",
    "data": {
        "message": "Session has started! You can now participate in voting."
    }
}
\end{lstlisting}

\subsubsection{User Presence Updates}

The system broadcasts user join/leave events:

\begin{lstlisting}[language=json, caption=User Joined Message]
{
    "type": "user_joined",
    "data": {
        "userId": "user-uuid",
        "userName": "Charlie"
    }
}
\end{lstlisting}

\begin{lstlisting}[language=json, caption=User Left Message]
{
    "type": "user_left",
    "data": {
        "userId": "user-uuid",
        "userName": "Charlie"
    }
}
\end{lstlisting}

\section{Security Considerations}

\subsection{Authentication and Authorization}

Currently, the system uses session-based user identification with the following security measures:

\begin{itemize}
    \item UUID-based session and user identification
    \item Role-based permission enforcement
    \item Server-side validation of all user actions
    \item Protection against unauthorized moderator actions
\end{itemize}

\subsection{Input Validation}

All client inputs undergo server-side validation:

\begin{itemize}
    \item Vote values restricted to valid Fibonacci sequence
    \item User names sanitized for display
    \item Message types validated against known protocols
    \item Session IDs validated for format and existence
\end{itemize}

\subsection{Future Security Enhancements}

Planned security improvements include:

\begin{itemize}
    \item JWT-based authentication
    \item Rate limiting for message frequency
    \item HTTPS enforcement
    \item Session timeout mechanisms
    \item Audit logging
\end{itemize}

\section{Deployment and Operations}

\subsection{Container Deployment}

The system supports Docker-based deployment with multi-stage builds for optimization:

\begin{lstlisting}[caption=Docker Configuration]
FROM golang:1.24-alpine AS builder
WORKDIR /app
COPY go.mod go.sum ./
RUN go mod download
COPY . .
RUN go build -o planning-poker

FROM alpine:latest
RUN apk --no-cache add ca-certificates
WORKDIR /root/
COPY --from=builder /app/planning-poker .
COPY web/ ./web/
EXPOSE 8080
CMD ["./planning-poker"]
\end{lstlisting}

\subsection{CI/CD Pipeline}

The project employs GitHub Actions for continuous integration and deployment:

\begin{itemize}
    \item Automated testing on Go 1.24+
    \item Static analysis with \texttt{go vet} and \texttt{staticcheck}
    \item Docker image building and testing
    \item Automated releases with semantic versioning
\end{itemize}

\section{Testing Strategy}

\subsection{Test Coverage}

The system implements comprehensive testing at multiple levels:

\begin{table}[H]
\centering
\begin{tabular}{@{}lp{8cm}@{}}
\toprule
\textbf{Test Type} & \textbf{Coverage} \\
\midrule
Unit Tests & Session management, user roles, voting logic \\
Integration Tests & WebSocket message flows, client-server interaction \\
End-to-End Tests & Complete user scenarios via test client \\
\bottomrule
\end{tabular}
\caption{Testing Coverage}
\label{tab:test-coverage}
\end{table}

\subsection{Test Client}

A dedicated Go test client simulates browser behavior for integration testing:

\begin{lstlisting}[language=go, caption=Test Client Usage]
// Create test clients
moderator := NewTestClient("Alice", sessionID, true)
participant := NewTestClient("Bob", sessionID, false)

// Connect and test workflow
moderator.Connect(serverURL)
participant.Connect(serverURL)
moderator.StartSession()
participant.Vote("5")
\end{lstlisting}

\section{Performance Considerations}

\subsection{Scalability}

Current design considerations for scalability:

\begin{itemize}
    \item In-memory session storage for low latency
    \item Efficient WebSocket connection management
    \item Minimal message overhead with JSON protocol
    \item Stateless server design for horizontal scaling
\end{itemize}

\subsection{Resource Usage}

Typical resource requirements:

\begin{itemize}
    \item Memory: ~10MB base + ~1KB per active user
    \item CPU: Minimal when idle, spikes during message broadcasts
    \item Network: ~1KB per message, scales with user count
\end{itemize}

\section{Infrastructure as Code}

\subsection{Overview}

The Planning Poker system includes comprehensive Infrastructure as Code (IaC) for automated deployment to Linode cloud infrastructure. The IaC implementation uses HashiCorp Packer for image building and Terraform for infrastructure provisioning, providing a fully automated path from source code to production deployment.

\subsection{Architecture Components}

The infrastructure deployment consists of two main components working in sequence:

\begin{enumerate}
    \item \textbf{Packer Image Builder}: Creates custom Linode images with Docker and deployment automation pre-installed
    \item \textbf{Terraform Infrastructure Provisioner}: Deploys the Packer image and configures networking, security, and application startup
\end{enumerate}

\begin{figure}[H]
\centering
\begin{lstlisting}[language=bash, caption=Infrastructure Deployment Flow]
# Step 1: Build custom image with Packer
Packer -> Ubuntu 22.04 Base Image
       -> Install Docker + Dependencies
       -> Copy Deployment Scripts
       -> Configure Systemd Service
       -> Create Custom Image

# Step 2: Deploy infrastructure with Terraform
Terraform -> Provision Linode Instance (Custom Image)
          -> Configure Firewall Rules
          -> Setup Networking
          -> Start Application Service
          -> Output Access Information
\end{lstlisting}
\end{figure}

\subsection{Packer Configuration}

\subsubsection{Base Image and Provisioning}

The Packer configuration builds upon Ubuntu 22.04 LTS, installing essential components for containerized application deployment:

\begin{table}[H]
\centering
\begin{tabular}{@{}ll@{}}
\toprule
\textbf{Component} & \textbf{Purpose} \\
\midrule
Docker CE & Container runtime for application deployment \\
Docker Compose & Multi-container application orchestration \\
GitHub CLI & Automated release artifact downloading \\
System Tools & htop, curl, wget, unzip, jq for administration \\
Deployment Script & Automated application startup and management \\
Systemd Service & System-level service management \\
\bottomrule
\end{tabular}
\caption{Packer Image Components}
\label{tab:packer-components}
\end{table}

\subsubsection{Deployment Automation}

The Packer image includes a deployment script (\texttt{/opt/planning-poker/deploy.sh}) that:

\begin{itemize}
    \item Fetches the latest release tag from GitHub API
    \item Downloads the Docker image from GitHub releases
    \item Loads the Docker image and starts the container
    \item Configures automatic restart policies
    \item Performs health checks and logging
\end{itemize}

\subsubsection{Service Configuration}

A systemd service (\texttt{planning-poker.service}) provides:

\begin{itemize}
    \item Automatic startup on system boot
    \item Dependency management (requires Docker service)
    \item Centralized logging via journald
    \item Service lifecycle management
\end{itemize}

\subsection{Terraform Configuration}

\subsubsection{Infrastructure Resources}

The Terraform configuration provisions the following Linode resources:

\begin{table}[H]
\centering
\begin{tabular}{@{}lll@{}}
\toprule
\textbf{Resource} & \textbf{Configuration} & \textbf{Purpose} \\
\midrule
Linode Instance & g6-nanode-1 (1GB RAM) & Application hosting \\
Cloud Firewall & SSH, HTTP, HTTPS, App ports & Network security \\
Domain Records & A and CNAME (optional) & DNS management \\
SSH Key & Public key authentication & Secure access \\
\bottomrule
\end{tabular}
\caption{Terraform Infrastructure Resources}
\label{tab:terraform-resources}
\end{table}

\subsubsection{Security Configuration}

The firewall configuration implements defense-in-depth principles:

\begin{lstlisting}[language=bash, caption=Firewall Rules Configuration]
# Inbound Rules (ACCEPT)
- SSH (port 22): Administrative access
- HTTP (port 80): Web traffic (redirect to HTTPS)
- HTTPS (port 443): Secure web traffic
- Application (port 8080): Planning Poker access

# Outbound Policy: ACCEPT (all outbound traffic allowed)
# Inbound Policy: DROP (default deny, explicit allow only)
\end{lstlisting}

\subsection{Prerequisites and Setup}

\subsubsection{Required Tools and Accounts}

\begin{table}[H]
\centering
\begin{tabular}{@{}lll@{}}
\toprule
\textbf{Requirement} & \textbf{Version} & \textbf{Purpose} \\
\midrule
Linode Account & Active with API access & Cloud infrastructure provider \\
Linode API Token & Full read/write permissions & Infrastructure provisioning \\
Packer & 1.9.0+ & Image building \\
Terraform & 1.5.0+ & Infrastructure provisioning \\
SSH Key Pair & RSA or Ed25519 & Server authentication \\
\bottomrule
\end{tabular}
\caption{Infrastructure Prerequisites}
\label{tab:infra-prerequisites}
\end{table}

\subsubsection{Linode API Token Configuration}

The Linode API token must have the following permissions:

\begin{itemize}
    \item \textbf{Linodes}: Read/Write (instance management)
    \item \textbf{Images}: Read/Write (custom image creation)
    \item \textbf{Domains}: Read/Write (DNS management, optional)
    \item \textbf{Firewall}: Read/Write (security rule management)
\end{itemize}

Generate the token at: \url{https://cloud.linode.com/profile/tokens}

\subsubsection{Environment Setup}

\begin{lstlisting}[language=bash, caption=Environment Configuration]
# Export Linode API token
export LINODE_TOKEN="your-linode-api-token-here"

# Initialize Packer plugins (first-time setup)
cd infrastructure/packer
packer init planning-poker.pkr.hcl

# Configure Terraform variables
cd ../terraform
cp terraform.tfvars.example terraform.tfvars

# Edit terraform.tfvars with:
# - linode_token: Your API token
# - root_password: Secure server password
# - ssh_public_key: Your public SSH key content
# - planning_poker_image: Packer image name
\end{lstlisting}

\subsection{Deployment Process}

\subsubsection{One-Command Deployment}

The infrastructure includes a Makefile for simplified operations:

\begin{lstlisting}[language=bash, caption=Automated Deployment]
# Complete deployment (Packer + Terraform)
cd infrastructure
make deploy

# Individual steps
make packer-build      # Build custom image
make terraform-apply   # Deploy infrastructure

# Management commands
make status            # Check application status
make show              # Display Terraform outputs
\end{lstlisting}

\subsubsection{Manual Deployment Steps}

For detailed control or troubleshooting:

\begin{lstlisting}[language=bash, caption=Manual Deployment Process]
# Step 1: Initialize and build Packer image
cd infrastructure/packer

# Initialize Packer (download required plugins)
packer init planning-poker.pkr.hcl

# Validate configuration
packer validate planning-poker.pkr.hcl

# Build the custom image
packer build planning-poker.pkr.hcl
packer build planning-poker.pkr.hcl

# Step 2: Configure Terraform
cd ../terraform
terraform init
terraform plan

# Step 3: Deploy infrastructure
terraform apply

# Step 4: Verify deployment
terraform output planning_poker_url
curl -I $(terraform output -raw planning_poker_url)
\end{lstlisting}

\subsection{Cost Analysis}

\subsubsection{Linode Pricing Structure}

\begin{table}[H]
\centering
\begin{tabular}{@{}lrrrl@{}}
\toprule
\textbf{Instance Type} & \textbf{RAM} & \textbf{CPU} & \textbf{Storage} & \textbf{Monthly Cost} \\
\midrule
g6-nanode-1 & 1GB & 1 vCPU & 25GB & \$5.00 \\
g6-standard-1 & 2GB & 1 vCPU & 50GB & \$10.00 \\
g6-standard-2 & 4GB & 2 vCPU & 80GB & \$20.00 \\
\bottomrule
\end{tabular}
\caption{Linode Instance Pricing (as of 2025)}
\label{tab:linode-pricing}
\end{table}

The default \texttt{g6-nanode-1} instance provides sufficient resources for small to medium Planning Poker sessions (10-50 concurrent users).

\subsubsection{Additional Costs}

\begin{itemize}
    \item \textbf{Bandwidth}: 1TB included, \$0.01/GB overage
    \item \textbf{Backups}: \$1.00/month (20\% of instance cost)
    \item \textbf{DNS}: \$0.00 (included with domain management)
    \item \textbf{Firewall}: \$0.00 (included service)
\end{itemize}

\subsection{Monitoring and Management}

\subsubsection{Application Health Monitoring}

\begin{lstlisting}[language=bash, caption=Health Check Commands]
# Application availability
curl -I http://instance-ip:8080

# Container status
ssh root@instance-ip 'docker ps'

# Service logs
ssh root@instance-ip 'journalctl -u planning-poker -f'

# System resources
ssh root@instance-ip 'htop'
\end{lstlisting}

\subsubsection{Update Management}

Application updates are automated through the deployment script:

\begin{lstlisting}[language=bash, caption=Update Process]
# Automatic update (fetches latest GitHub release)
ssh root@instance-ip 'systemctl restart planning-poker'

# Manual update to specific version
ssh root@instance-ip
# Edit /opt/planning-poker/deploy.sh to specify version
systemctl restart planning-poker
\end{lstlisting}

\subsection{Security Considerations}

\subsubsection{Network Security}

\begin{itemize}
    \item Firewall-based access control with minimal port exposure
    \item SSH key-based authentication (password authentication disabled)
    \item Regular security updates via automated package management
    \item Network isolation through Linode's private networking
\end{itemize}

\subsubsection{Application Security}

\begin{itemize}
    \item Container isolation through Docker
    \item Non-root container execution
    \item Automated security updates for base image
    \item Regular image rebuilds with latest security patches
\end{itemize}

\subsubsection{Access Control}

\begin{itemize}
    \item SSH access restricted to authorized public keys
    \item Root access protected by strong password + SSH keys
    \item Application runs in unprivileged container context
    \item Firewall rules limit attack surface
\end{itemize}

\subsection{Production Deployment}

\subsubsection{Successful IaC Implementation}

The Infrastructure as Code deployment has been successfully validated and deployed to production on Linode cloud infrastructure. The deployment demonstrates the complete automation pipeline from source code to running application.

\begin{table}[H]
\centering
\begin{tabular}{@{}ll@{}}
\toprule
\textbf{Component} & \textbf{Status} \\
\midrule
Packer Image Build & \textcolor{green}{\checkmark} Completed Successfully \\
Terraform Infrastructure & \textcolor{green}{\checkmark} Deployed and Running \\
Application Health & \textcolor{green}{\checkmark} Passing All Checks \\
Security Configuration & \textcolor{green}{\checkmark} Firewall Rules Active \\
DNS Resolution & \textcolor{green}{\checkmark} Accessible via IP \\
Container Runtime & \textcolor{green}{\checkmark} Docker Service Running \\
\bottomrule
\end{tabular}
\caption{Production Deployment Status}
\label{tab:deployment-status}
\end{table}

\subsubsection{Production Environment Details}

The current production deployment configuration:

\begin{lstlisting}[language=bash, caption=Production Infrastructure Configuration]
# Infrastructure Details
Region: us-east (Linode)
Instance Type: g6-nanode-1 (1GB RAM, 1 vCPU, 25GB SSD)
Monthly Cost: $5.00 USD

# Network Configuration  
Public IP: 172.104.9.12
Firewall ID: 2942935
Instance ID: 79489141

# Application Access
Application URL: http://172.104.9.12:8080
SSH Access: ssh root@172.104.9.12
Container: planning-poker:v0.8.2

# Service Status
Planning Poker Service: Active (running)
Docker Container: Running (auto-restart enabled)
Health Checks: Passing
Uptime: Since deployment
\end{lstlisting}

\subsubsection{Deployment Validation Results}

The production deployment has been thoroughly tested and validated:

\begin{table}[H]
\centering
\begin{tabular}{@{}lll@{}}
\toprule
\textbf{Test Category} & \textbf{Test Case} & \textbf{Result} \\
\midrule
Infrastructure & Packer image creation & \textcolor{green}{PASS} \\
Infrastructure & Terraform resource provisioning & \textcolor{green}{PASS} \\
Infrastructure & Firewall rule application & \textcolor{green}{PASS} \\
Application & HTTP endpoint accessibility & \textcolor{green}{PASS} \\
Application & Docker container health & \textcolor{green}{PASS} \\
Application & Service auto-start capability & \textcolor{green}{PASS} \\
Security & SSH key authentication & \textcolor{green}{PASS} \\
Security & Port access restrictions & \textcolor{green}{PASS} \\
Automation & GitHub Actions validation & \textcolor{green}{PASS} \\
Documentation & IaC process documentation & \textcolor{green}{PASS} \\
\bottomrule
\end{tabular}
\caption{Production Deployment Validation Matrix}
\label{tab:deployment-validation}
\end{table}

\subsubsection{Operational Commands}

Essential commands for production environment management:

\begin{lstlisting}[language=bash, caption=Production Management Commands]
# Application Health Check
curl -I http://172.104.9.12:8080

# Service Management
ssh root@172.104.9.12 'systemctl status planning-poker'
ssh root@172.104.9.12 'systemctl restart planning-poker'

# Container Management  
ssh root@172.104.9.12 'docker ps'
ssh root@172.104.9.12 'docker logs planning-poker'

# Infrastructure Management
terraform show                    # View current state
terraform plan                    # Preview changes
terraform output                  # Show deployment outputs

# Image Management
packer build planning-poker.pkr.hcl  # Rebuild image
\end{lstlisting}

\subsubsection{Lessons Learned}

Key insights from the production deployment process:

\begin{itemize}
    \item \textbf{SSH Key Management}: Terraform provisioners require passwordless SSH keys. Passphrase-protected keys must be handled differently or provisioning logic moved to Packer images.
    \item \textbf{Image Dependency}: Custom Packer images significantly reduce deployment complexity by pre-installing dependencies and deployment scripts.
    \item \textbf{Health Monitoring}: Automated health checks in deployment scripts provide early failure detection and improve deployment reliability.
    \item \textbf{Security by Design}: Implementing firewall rules and SSH key authentication from the start prevents security gaps during deployment.
    \item \textbf{Documentation Importance}: Comprehensive documentation of the IaC process enables reproducible deployments and team knowledge sharing.
\end{itemize}

\section{Future Enhancements}

\subsection{Planned Features}

\begin{itemize}
    \item Persistent session storage (Redis/PostgreSQL)
    \item Custom voting scales
    \item Session analytics and reporting
    \item Mobile-responsive design improvements
    \item Integration with project management tools
\end{itemize}

\subsection{Architecture Evolution}

Long-term architectural considerations:

\begin{itemize}
    \item Microservices decomposition
    \item Event-driven architecture with message queues
    \item Multi-region deployment support
    \item Real-time analytics dashboard
\end{itemize}

\section{Conclusion}

The Planning Poker system successfully implements a robust, real-time collaborative estimation platform suitable for distributed Agile teams. The WebSocket-based architecture provides low-latency communication while maintaining simplicity and reliability.

The system's modular design and comprehensive testing strategy ensure maintainability and extensibility for future enhancements. The documented message protocols and security considerations provide a solid foundation for integration and deployment in production environments.

\bibliographystyle{plain}
\bibliography{references}

\end{document}
